\documentclass[10pt,a4paper]{article}
\usepackage[T2A]{fontenc}    
\usepackage[english]{babel}   
\usepackage{color}
\usepackage[urlbordercolor={1 1 1},colorlinks=true]{hyperref}
\usepackage{longtable}
\usepackage{graphicx}
\usepackage[a4paper,left=2.5cm,right=2cm,top=2cm,bottom=2cm]{geometry}

\usepackage{mathptmx}
\usepackage{arev}
\usepackage{indentfirst}
% \usepackage[default,osfigures,scale=0.95]{opensans}
% \usepackage[defaultsans]{droidsans}
% \usepackage{fontspec}
% \renewcommand{\sfdefault}{phv}
% \renewcommand{\sfdefault}{PTSansCaption-TLF}
% \fontfamily{pag}\selectfont
% \renewcommand{\rmdefault}{\sfdefault}
% \renewcommand{\rmdefault}{ptm}
% \setmainfont{Times}
% \fontfamily{Name_OF_Font_Family}\selectfont


%\input{newcommands}
\definecolor{link}{rgb}{0,0,0.6}

\hypersetup
{
	linkcolor=link,
	urlcolor={link}
}

\newcommand{\lmpratio}{0.15}
\newcommand{\rmpratio}{0.74}
\newcommand{\verticalSpace}{0.3cm}
\newcommand{\vSpace}{0.5cm}
\newcommand{\horizontalSpace}{0.05\textwidth}

\newcommand{\sectionTitle}[1]{\Large{\textbf{#1}}}
\newcommand{\sectionMain}[1]{\textbf{#1}}

\newcommand{\vacancyName}{PhD student}
\renewcommand*\familydefault{\sfdefault}
% \renewcommand{\familydefault}{\sfdefault}


\setlength{\parindent}{3em}
\setlength{\parskip}{0.5em}

\begin{document}

	\pagenumbering{gobble} 
	

	\raggedright{\Large{\textbf{Arseniy Shchepetnov}}}\\[0.3cm]
	
	\begin{minipage}[t]{0.7\textwidth}
		\vspace{0pt}
		\raggedright{\textbf{Lead Data Scientist}}\\[0.3cm]
		% 	\raggedright{\huge\vacancyName}\\[0.3cm]
		% 	\noindent Date of birth: $15^{\mathrm{th}}$ December $1989$ \\[0.1cm]
		\noindent Address: Saint-Petersburg, Russian Federation \\[0.1cm]
		\noindent Email: \href{mailto:a.shchepetnov@gmail.com}{a.shchepetnov@gmail.com}\\[0.1cm]
		\noindent Phone: $+7\,953\,149\,42\,68$ \\
		\noindent GitHub: \href{https://github.com/ArseniyShchepetnov}{ArseniyShchepetnov}\\
            \noindent LinkedIn: \href{https://www.linkedin.com/in/arseniy-shchepetnov-4a236171/}{Arseniy Shchepetnov}\\
            \noindent Medium: \href{https://medium.com/@a.shchepetnov}{Arseniy Shchepetnov}
	\end{minipage}
	\begin{minipage}[t]{0.2\textwidth}
		\vspace{0pt}
		% \includegraphics[width=\linewidth]{photo.png}
	\end{minipage}
	
	
	\section*{Objective}
	
	Data Science/Lead Data Science, Development
	
	\section*{Relevant skills}
	
	\setlength{\parindent}{3em}

	
My interest vector is referred mostly to Data Science, R\&D, Product Development related to technologies and research.

I have advanced development skills with Python including linting and CI/CD (see examples on GitHub).
Other languages that are covered with my experience are C++, Fortran, R.

In different organizations I have developed multiple algorithms and product components that were successfully deployed.
Under my direction new data science service for was developed related to geological data analysis with deep learning models.
I like to work in team and learn new things, and my progress was marked by collegues with multiple promotions.

	
	
	
	
	
	\setlength{\parindent}{0em}
	\vspace{\verticalSpace}
%-------------------------------------------------------------------------------------------------------------------------------	
	\vspace{\verticalSpace}
	\section*{Employment history}

	% TEDO
	\begin{minipage}[t]{\lmpratio\textwidth}
		July 2022 --- Now
	\end{minipage}
	\hspace{\horizontalSpace}
	\begin{minipage}[t]{\rmpratio\textwidth}
		\sectionMain{Lead Data Scientist/Team Lead/Manager}\\
		\href{https://www.tedo.ru/}{<<Digital Trust Formula>>} (Former <<PWC>> unit) (Artificial Intelligence Team), Saint-Petersburg\\[0.3cm]		
		
Resumption of the projects from <<PWC>> due to rebranding of the former <<PWC>> unit.
		
			
	\end{minipage}	
	\vspace{\vSpace}

	% PWC
	\begin{minipage}[t]{\lmpratio\textwidth}
		May 2021 --- \\July 2022
	\end{minipage}
	\hspace{\horizontalSpace}
	\begin{minipage}[t]{\rmpratio\textwidth}
		\sectionMain{Lead Data Scientist/Team Lead/Senior Consultant}\\
		\href{https://www.pwc.ru/}{<<PWC>>} (Technology, Artificial Intelligence), Saint-Petersburg\\[0.3cm]		
		
Research and Development in ``AI Practice'' in the Technology Consulting unit.\\
Leading team with 3 Junior Data Scientists.\\
		
Multiple projects were carried out simultaneously.
		
		\begin{itemize}
			\item Numerical service development (FastAPI) with Template-based Bayesian Network as a part of the cognitive product. 
			During the project hypothesis construction and testing, bayesian network module optimization were performed. 
			MVP was launched successfully and further product development in the part of the numerical module.
			\item Machine Learning and Deep Learning well logs data analysis in order to predict productive saturation as a resumption of the project started at IBM.
			Development of well logs correlation methods.
			\item ``Proof of principle'' in the field of well logs recovery with machine and deep learning techniques.
			\item Participation in neighboring projects to support hiring or technology consulting.
			
		\end{itemize}
		
		Also, I designed Objective DataFrame framework for pandas DataFrame's serving.
		
		
		
	\end{minipage}	
	\vspace{\vSpace}
	
	% IBM
	\begin{minipage}[t]{\lmpratio\textwidth}
		May 2018 --- \\May 2021
	\end{minipage}
	\hspace{\horizontalSpace}
	\begin{minipage}[t]{\rmpratio\textwidth}
		\sectionMain{Data Scientist/Lead Data Scientist/Team Lead}\\
		\href{https://www.ibm.com/ru/rstl/index-en.html}{<<IBM STC>>} (GBS), Saint-Petersburg, Moscow\\[0.5cm]		
		Work with ``Cognitive Practice Team'' in the consulting Global Business Service unit in IBM.
		
		Main duty was to create ML solutions for Oil\&{}Gas problems. From year 2020 started leading 2 Junior Data Scientist.
		

		The main project was dedicated to saturation prediction, well logs correlation and product development with the features with deep learning and machine learning numerical modules. Model serving solutions were developed on AWS including model library, datamarts, pipelines and multiple technologies were used.
				
				
		Project resulted in the MVP and some productive intervals (oil) predicted by the system were successfully collected from real wells.
		
		
	\end{minipage}	
	\vspace{\vSpace}
	
	% Healbe
	\begin{minipage}[t]{\lmpratio\textwidth}
		Aug 2016 --- \\May 2018
	\end{minipage}
	\hspace{\horizontalSpace}
	\begin{minipage}[t]{\rmpratio\textwidth}
		\sectionMain{Research and Development}\\
		\href{https://healbe.com/}{<<Healbe corp.>>}, Saint-Petersburg\\[0.5cm]		
		
		 Development of algorithms for wearable devices signals processing. 
		 Successfully developed algorithms for custom sensor and for optical heart rate sensor.
Team leader position was offered to me.
		 
		
	\end{minipage}	
	\vspace{\vSpace}
	
	% SPBSU	
	\begin{minipage}[t]{\lmpratio\textwidth}
		Jan 2015 --- \\Dec 2015
	\end{minipage}
	\hspace{\horizontalSpace}
	\begin{minipage}[t]{\rmpratio\textwidth}
		\sectionMain{Researcher}\\
		\href{http://english.spbu.ru/}{<<Saint-Petersburg State University>>}, Saint-Petersburg\\[0.5cm]		
		Research in the field of atomic physics. 
		Main themes: $g$ factor theory, nuclear recoil effect. \\

	\end{minipage}
	\vspace{\vSpace}

	\begin{minipage}[t]{\lmpratio\textwidth}
		Jan 2014 --- \\Dec 2016
	\end{minipage}
	\hspace{\horizontalSpace}
	\begin{minipage}[t]{\rmpratio\textwidth}
		\sectionMain{Researcher}\\
		\href{http://frrc.itep.ru/index.php/en/}{<<Institute for Theoretical and Experimental Physics>>}, Moscow\\[0.5cm]
		Research in the field of atomic physics supported by ``Helmholtz-Rosatom'' grant. 
		Theme: ``Zeeman splitting in highly-charged ions: novel approach to the non-linear effects''. \\[0.5cm]
		
	\end{minipage}
	
	\vspace{\vSpace}

	\begin{minipage}[t]{\lmpratio\textwidth}
		Nov 2013 --- \\Feb 2016
	\end{minipage}
	\hspace{\horizontalSpace}
	\begin{minipage}[t]{\rmpratio\textwidth}
		\sectionMain{Engineer-programmer}\\
		\href{http://www.rimr.ru/eng/}{<<Russian Institute for Power Radiobuilding>>}, Saint-Petersburg\\[0.5cm]
		Linux based software development for HF radio equipment: C/C++, Qt4, PostgreSQL.
		Designed and developed database, GUI and successfully deployed.
	\end{minipage}

		\vspace{\vSpace}
	
	\begin{minipage}[t]{\lmpratio\textwidth}
		Sep 2011 --- \\Jan 2012
	\end{minipage}
	\hspace{\horizontalSpace}
	\begin{minipage}[t]{\rmpratio\textwidth}
		\sectionMain{Teacher (Physics)}\\
		\href{http://lnmo.ru/}{<<Laboratory for Continuous Mathematical Education>>}, Saint-Petersburg\\[0.5cm]
		Teaching physics at 8 and 9 year classes. Special seminars on thermodynamics.
	\end{minipage}	
	\vspace{\verticalSpace}
%-------------------------------------------------------------------------------------------------------------------------------
	\vspace{\verticalSpace}
	\section*{Education}
	
	\begin{minipage}[t]{\lmpratio\textwidth}
		Sep 2013 --- \\Jul 2016
	\end{minipage}
	\hspace{\horizontalSpace}
	\begin{minipage}[t]{\rmpratio\textwidth}
		\sectionMain{Postgraduate student} (Theoretical Physics)\\[0.1cm]		
		\href{http://english.spbu.ru/}{Saint-Petersburg State University}\\ Department of Physics, Division of Quantum Mechanics\\[0.3cm]
		 Thesis: ``Nuclear recoil corrections to the $g$ factor of highly-charged ions'' \\[0.3cm]
		 Scientific advisors: \href{http://fock.phys.spbu.ru/english/tupicin_en.htm}{Prof. Ilya Tupitsyn} and \href{http://fock.phys.spbu.ru/glazov.htm}{Dr. Dmitry Glazov}
	\end{minipage}

	\vspace{1cm}
	
	\begin{minipage}[t]{\lmpratio\textwidth}
		Sep 2011 --- \\Jul 2013
	\end{minipage}
	\hspace{\horizontalSpace}
	\begin{minipage}[t]{\rmpratio\textwidth}
		\sectionMain{Master's degree} (Quantum Mechanics of Atoms, Molecules and Solids) \\[0.1cm]
		\href{http://english.spbu.ru/}{Saint-Petersburg State University}\\ Department of Physics, Division of Quantum Mechanics\\[0.3cm]
		 Thesis: ``Nuclear recoil corrections to the energy levels and to the $g$ factor of highly-charged ions''\\[0.3cm]
		 Scientific advisor: \href{http://fock.phys.spbu.ru/english/tupicin_en.htm}{Prof. Ilya Tupitsyn} and \href{http://fock.phys.spbu.ru/glazov.htm}{Dr. Dmitry Glazov}
	\end{minipage}

	\vspace{1cm}
	
	\begin{minipage}[t]{\lmpratio\textwidth}
		Sep 2007 --- \\Jul 2011
	\end{minipage}
	\hspace{\horizontalSpace}
	\begin{minipage}[t]{\rmpratio\textwidth}
		\sectionMain{Bachelor's degree} (Quantum Mechanics of Atoms, Molecules and Solids) \\[0.1cm]
		\href{http://english.spbu.ru/}{Saint-Petersburg State University}\\ Department of Physics, Division of Quantum Mechanics\\[0.3cm]
		 Thesis: ``Nuclear recoil corrections to the energy levels of highly-charged ions''\\[0.3cm]
		 Scientific advisor: \href{http://fock.phys.spbu.ru/english/shabaev_en.htm}{Prof. Vladimir Shabaev}
	\end{minipage}
	
	\newpage
	
%-------------------------------------------------------------------------------------------------------------------------------
	\section*{Certificates}	

        \begin{minipage}[t]{\lmpratio\textwidth}
		Coursera\\August 2022
	\end{minipage}
	\hspace{\horizontalSpace}
	\begin{minipage}[t]{\rmpratio\textwidth}
		\sectionMain{Natural Language Processing Specialization}\\
            \href{https://www.coursera.org/account/accomplishments/specialization/certificate/D4F2WRLEHCY8}{D4F2WRLEHCY8}
	\end{minipage}
	\vspace{1cm}

        \begin{minipage}[t]{\lmpratio\textwidth}
		Coursera\\August 2022
	\end{minipage}
	\hspace{\horizontalSpace}
	\begin{minipage}[t]{\rmpratio\textwidth}
		\sectionMain{Natural Language Processing with Attention Models}\\
		\href{https://www.coursera.org/account/accomplishments/certificate/TT5J2NHDYDY2}{TT5J2NHDYDY2}
	\end{minipage}
	\vspace{1cm}

	\begin{minipage}[t]{\lmpratio\textwidth}
		Coursera\\August 2022
	\end{minipage}
	\hspace{\horizontalSpace}
	\begin{minipage}[t]{\rmpratio\textwidth}
		\sectionMain{Natural Language Processing with Classification and Vector Spaces}\\
		\href{https://www.coursera.org/account/accomplishments/certificate/3LL65SQH6EWM}{3LL65SQH6EWM}
	\end{minipage}
	\vspace{1cm}

	\begin{minipage}[t]{\lmpratio\textwidth}
		Coursera\\August 2022
	\end{minipage}
	\hspace{\horizontalSpace}
	\begin{minipage}[t]{\rmpratio\textwidth}
		\sectionMain{Natural Language Processing with Probabilistic Models}\\
		\href{https://www.coursera.org/account/accomplishments/certificate/C3RSRG7T853E}{C3RSRG7T853E}
	\end{minipage}
	\vspace{1cm}
	

	\begin{minipage}[t]{\lmpratio\textwidth}
		Coursera\\August 2022
	\end{minipage}
	\hspace{\horizontalSpace}
	\begin{minipage}[t]{\rmpratio\textwidth}
		\sectionMain{Natural Language Processing with Sequence Models}\\
		\href{https://www.coursera.org/account/accomplishments/certificate/667D47FVSK3M}{667D47FVSK3M}
	\end{minipage}
	\vspace{1cm}


	\begin{minipage}[t]{\lmpratio\textwidth}
		Coursera\\July 2022
	\end{minipage}
	\hspace{\horizontalSpace}
	\begin{minipage}[t]{\rmpratio\textwidth}
		\sectionMain{Combinatorics and Probability}\\
		\href{https://www.coursera.org/account/accomplishments/certificate/EFYQQQ9GUTUP}{EFYQQQ9GUTUP}
	\end{minipage}
	\vspace{1cm}

	\begin{minipage}[t]{\lmpratio\textwidth}
		Coursera\\June 2022
	\end{minipage}
	\hspace{\horizontalSpace}
	\begin{minipage}[t]{\rmpratio\textwidth}
		\sectionMain{Sample-based Learning Methods}\\
		\href{https://www.coursera.org/account/accomplishments/certificate/KC4T942AATVE}{KC4T942AATVE}
	\end{minipage}
	\vspace{1cm}

		
	\begin{minipage}[t]{\lmpratio\textwidth}
		Coursera\\May 2022
	\end{minipage}
	\hspace{\horizontalSpace}
	\begin{minipage}[t]{\rmpratio\textwidth}
		\sectionMain{Fundamentals of Reinforcement Learning}\\
		\href{https://www.coursera.org/account/accomplishments/certificate/U3NT5V7XZCNK}{U3NT5V7XZCNK}
	\end{minipage}
	\vspace{1cm}
		
	
%-------------------------------------------------------------------------------------------------------------------------------
	\section*{Publications}
	\begin{itemize}
		\item A.~A.~Shchepetnov, D.~A.~Glazov, A.~V.~Volotka, V.~M.~Shabaev, I.~I.~Tupitsyn, G.~Plunien 
			``Nuclear recoil correction to the $g$ factor of boron-like argon''. Journal of Physics: Conference Series, 2015. --- Vol. 583, --- P. 012001
		\item I.~A.~Aleksandrov, A.~A.~Shchepetnov, D.~A.~Glazov, V.~M.~Shabaev 
			``Finite nuclear size corrections to the recoil effect in hydrogenlike ions''. Journal of Physics B: Atomic, Molecular and Optical Physics, 2015. --- Vol. 48, --- No. 14. --- P. 144004		
		\item D.~A.~Glazov, A.~V.~Volotka, A.~A.~Schepetnov, M.~M.~Sokolov, V.~M.~Shabaev, I.~I.~Tupitsyn, G.~Plunien 
			``$g$ factor of boron-like ions: ground and excited states''. Physica Scripta, 2013. --- Vol. T156, --- P. 014014
	\end{itemize}
% 	New articles are in preparation. 
% 	\vspace{\verticalSpace}
% 	\newpage
	\section*{International conferences}
		\begin{itemize}
			\item[---]	Seminar on Astrophysics, Clocks and Fundamental Constants, Bad Honnef, Germany (Poster), 2015
			\item[---]	Topical Workshop of the SPARC Collaboration, Worms, Germany (Poster), 2014
			\item[---]	\href{http://indico.gsi.de/conferenceDisplay.py?confId=2443}{International Conference on Science and Technology for FAIR}, Worms, Germany (Poster), 2014
			\item[---]	\href{http://www.cab.cnea.gov.ar/hci2014/}{International Conference on the Physics of Highly Charged Ions}, San Carlos de Bariloche, Argentina (\textbf{Oral presentation}), 2014
			\item[---]	\href{https://sites.google.com/site/finqinternational/home}{Fin Q workshop}: International conference <<Quantum Informatics and Applications>>, St. Petersburg, Russia (No report), 2013
			\item[---]	\href{http://www.gao.spb.ru/russian/psas/ffk2013/}{The Workshop on Precision Physics and Fundamental Physical Constants}, Pulkovo, St. Petersburg, Russia (Poster), 2013
			\item[---]	\href{http://fas.vsu.ru/en/index.php}{XX Conference on Fundamental Atomic Spectroscopy}, Voronezh, Russia (\textbf{Oral presentation}), 2013
		\end{itemize}		
	
	\vspace{5cm}
	
	Arseniy Shchepetnov
	
	
	
	
\end{document}

